% -*- coding: utf-8 -*-
% !TEX program = lualatex
\documentclass[oneside]{book}

% -*- coding: utf-8 -*-
% !TEX program = lualatex

\newcommand*{\myversion}{2025A}
\newcommand*{\mylpad}[1]{\ifnum#1<10 0\the#1\else\the#1\fi}

\usepackage[a4paper,margin=2.5cm]{geometry}

\setlength{\parindent}{0pt}
\setlength{\parskip}{4pt plus 1pt minus 1pt}

\makeatletter
\ExplSyntaxOn
\NewDocumentCommand\MyDebugSingle{m}{
  \@ifpackageloaded{docmute}{#1}{
    \str_if_eq:onTF{\jobname}{tabularray}{#1}{
      \sys_ensure_backend:
      \debug_on:n{check-declarations}
      #1
      \debug_off:n{check-declarations}
    }
  }
}
\ExplSyntaxOff
\makeatother
\MyDebugSingle{\usepackage{tabularray}}

\usepackage{codehigh} % https://ctan.org/pkg/codehigh
\usepackage{array,multirow,amsmath}
\usepackage{chemmacros,environ}
\usepackage{enumitem}

\usepackage[firstpage=true]{background}
\backgroundsetup{contents={}}

\UseTblrLibrary{
  amsmath,booktabs,counter,diagbox,functional,siunitx,tikz,varwidth
}
\usetikzlibrary{patterns}

\usepackage{hyperref}
\hypersetup{
  colorlinks=true,
  urlcolor=blue3,
  linkcolor=blue3,
}

\usepackage{tcolorbox}
\tcbset{sharp corners, boxrule=0.5pt, colback=red9}

\usepackage{float}
%\usepackage{enumerate}

\setcounter{tocdepth}{1}

\NewDocumentCommand\None{}{{\boldmath$\times$}}
\NewDocumentCommand\K{m}{\texttt{#1}} % key
\NewDocumentCommand\V{m}{\texttt{#1}} % value
\NewDocumentCommand\Q{m}{\texttt{#1}} % column/row type

\NewDocumentCommand\KK{m}{\texttt{\fakeverb{#1}}} % key
\NewDocumentCommand\VV{m}{\texttt{\fakeverb{#1}}} % value
\NewDocumentCommand\KV{m}{\texttt{\fakeverb{#1}}} % key and value
\NewDocumentCommand\CC{m}{\texttt{\fakeverb{#1}}} % command
\NewDocumentCommand\EE{m}{\texttt{\fakeverb{#1}}} % environment
\NewDocumentCommand\LL{m}{\texttt{\fakeverb{#1}}} % library
\NewDocumentCommand\PP{m}{\texttt{\fakeverb{#1}}} % package
\NewDocumentCommand\FF{m}{\texttt{\fakeverb{#1}}} % file
\NewDocumentCommand\NN{m}{\texttt{\fakeverb{#1}}} % tikz node
\NewDocumentCommand\KP{m}{\texttt{\fakeverb{#1}}} % key path
\NewDocumentCommand\HP{m}{\texttt{\fakeverb{#1}}} % hook path
\NewDocumentCommand\CI{m}{\texttt{\fakeverb{#1}}} % child indexer
\NewDocumentCommand\CS{m}{\texttt{\fakeverb{#1}}} % child selector
\NewDocumentCommand\CO{m}{\texttt{\fakeverb{#1}}} % counter
\NewDocumentCommand\EN{m}{\texttt{\fakeverb{#1}}} % element name
\NewDocumentCommand\TN{m}{\texttt{\fakeverb{#1}}} % template name
\NewDocumentCommand\TT{m}{\texttt{\fakeverb{#1}}} % text

\NewTblrEnviron{newtblr}
\SetTblrOuter[newtblr]{long}
\SetTblrInner[newtblr]{
  hlines = {gray3}, column{1,2} = {co=1}, colsep = 5pt,
  row{2-Z} = {brown8},
  row{1} = {fg=white, bg=purple2, font=\bfseries\sffamily},
}

\NewTblrEnviron{spectblr}
\SetTblrOuter[spectblr]{long}
\SetTblrInner[spectblr]{
  hlines = {gray3}, column{2} = {co=1}, colsep = 5pt,
  row{2-Z} = {brown8},
  row{1} = {fg=white, bg=purple2, font=\bfseries\sffamily},
  rowhead = 1,
}

\renewcommand\emph[1]{\textit{\color{red3}#1}}

\newcommand{\mywarning}[1]{%
  \begin{tcolorbox}
  #1
  \end{tcolorbox}%
}

%\renewcommand*{\thefootnote}{*}

\colorlet{highback}{azure9}
\CodeHigh{language=latex/table,style/main=highback,style/code=highback}
\NewCodeHighEnv{code}{style/main=gray9,style/code=gray9}
\NewCodeHighEnv{demo}{style/main=gray9,style/code=gray9,demo}

%\CodeHigh{lite}

\CodeHigh{lite}
\setcounter{chapter}{5}

\begin{document}

\chapter{Tips and Tricks}

\section{Default rule widths and colors}

From version 2025A, default hrule and vrule widths are stored in variables
\CC{\lTblrDefaultHruleWidthDim} and \CC{\lTblrDefaultVruleWidthDim} respectively,
and default hrule and vrule colors are stored in variables \CC{\lTblrDefaultHruleColorTl}
and \CC{\lTblrDefaultVruleColorTl} respectively. Here is an example:

\begin{demohigh}
\setlength\lTblrDefaultHruleWidthDim{1pt}%
\setlength\lTblrDefaultVruleWidthDim{2pt}%
\renewcommand\lTblrDefaultHruleColorTl{blue5}%
\renewcommand\lTblrDefaultVruleColorTl{red5}%
\begin{tblr}{
  hlines, hline{2} = {wd=2pt, fg=cyan5},
  vlines, vline{2} = {wd=1pt, fg=green5}
}
  Alpha   & Beta  & Gamma  \\
  Epsilon & Zeta  & Eta    \\
  Iota    & Kappa & Lambda \\
\end{tblr}
\end{demohigh}

\section{Control horizontal alignment}

You can control horizontal alignment of cells in \texttt{tabularray} with
\href{https://www.ctan.org/pkg/ragged2e}{\texttt{ragged2e}} package,
by redefining some of the following commands:

\begin{codehigh}
\RenewDocumentCommand\TblrAlignBoth{}{\justifying}
\RenewDocumentCommand\TblrAlignLeft{}{\RaggedRight}
\RenewDocumentCommand\TblrAlignCenter{}{\Centering}
\RenewDocumentCommand\TblrAlignRight{}{\RaggedLeft}
\end{codehigh}

Please read the documentation of \texttt{ragged2e} package for more details of
their alignment commands.

\section{Use safe verbatim commands}%
\label{sec:fakeverb}

Due to the limitations of TeX,
we are not able to make \CC{\verb} command behave well inside \PP{tabularray} tables.
As a replacement, you may use \CC{\fakeverb} command from \href{https://www.ctan.org/pkg/codehigh}{\PP{codehigh}} package.

The \CC{\fakeverb} command will remove the backslashes in the following control symbols before
typesetting its content: \CC{\\\\}, \CC{\\\{}, \CC{\\\}}, \CC{\\\#}, \CC{\\\^} and \texttt{\textbackslash\textvisiblespace}, \CC{\\\%}.
Also the argument of \CC{\fakeverb} command need to be enclosed with curly braces.
Therefore it could be safely used inside \PP{tabularray} tables and other LaTeX commands.

Here is an example of using \CC{\fakeverb} commands inside a \EE{tblr} environment:

\begin{demohigh}
\begin{tblr}{hlines}
  Special & \fakeverb{\abc{}$&^_^uvw 123} \\
  Spacing & \fakeverb{\bfseries\ \#\%}    \\
  Nesting & \fbox{\fakeverb{$\left\\\{A\right.$\#}}
\end{tblr}
\end{demohigh}

In the above example, balanced curly braces and control words (such as \CC{\bfseries})
need not to be escaped---only several special characters need to be escaped.
Please read the documentation of \PP{codehigh} package for more details of
\CC{\fakeverb} commands.%
\footnote{By the way, \CC{\EscVerb} command from
\href{https://www.ctan.org/pkg/fvextra}{\PP{fvextra}} package is similar to
\CC{\fakeverb} command, but with \CC{\EscVerb} you need to escape every control word.}

\section{Blank lines around cells}

In \PP{tabularray} tables, there could be a blank line before a cell, after a cell,
or between table commands and cell text. Here is an example:

\begin{demohigh}
\begin{tblr}{rl}

\hline

  One

  &

  Two

  \\

\hline

  Three

  &

  Four

  \\

\hline

\end{tblr}
\end{demohigh}

But more blank lines are not supported.
Therefore putting more than one blank line at any of these positions may cause wrong result.


\end{document}
