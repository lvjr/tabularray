% -*- coding: utf-8 -*-
% !TEX program = lualatex
\documentclass[oneside]{book}

% -*- coding: utf-8 -*-
% !TEX program = lualatex

\newcommand*{\myversion}{2025A}
\newcommand*{\mylpad}[1]{\ifnum#1<10 0\the#1\else\the#1\fi}

\usepackage[a4paper,margin=2.5cm]{geometry}

\setlength{\parindent}{0pt}
\setlength{\parskip}{4pt plus 1pt minus 1pt}

\makeatletter
\ExplSyntaxOn
\NewDocumentCommand\MyDebugSingle{m}{
  \@ifpackageloaded{docmute}{#1}{
    \str_if_eq:onTF{\jobname}{tabularray}{#1}{
      \sys_ensure_backend:
      \debug_on:n{check-declarations}
      #1
      \debug_off:n{check-declarations}
    }
  }
}
\ExplSyntaxOff
\makeatother
\MyDebugSingle{\usepackage{tabularray}}

\usepackage{codehigh} % https://ctan.org/pkg/codehigh
\usepackage{array,multirow,amsmath}
\usepackage{chemmacros,environ}
\usepackage{enumitem}

\usepackage[firstpage=true]{background}
\backgroundsetup{contents={}}

\UseTblrLibrary{
  amsmath,booktabs,counter,diagbox,functional,siunitx,tikz,varwidth
}
\usetikzlibrary{patterns}

\usepackage{hyperref}
\hypersetup{
  colorlinks=true,
  urlcolor=blue3,
  linkcolor=blue3,
}

\usepackage{tcolorbox}
\tcbset{sharp corners, boxrule=0.5pt, colback=red9}

\usepackage{float}
%\usepackage{enumerate}

\setcounter{tocdepth}{1}

\NewDocumentCommand\None{}{{\boldmath$\times$}}
\NewDocumentCommand\K{m}{\texttt{#1}} % key
\NewDocumentCommand\V{m}{\texttt{#1}} % value
\NewDocumentCommand\Q{m}{\texttt{#1}} % column/row type

\NewDocumentCommand\KK{m}{\texttt{\fakeverb{#1}}} % key
\NewDocumentCommand\VV{m}{\texttt{\fakeverb{#1}}} % value
\NewDocumentCommand\KV{m}{\texttt{\fakeverb{#1}}} % key and value
\NewDocumentCommand\CC{m}{\texttt{\fakeverb{#1}}} % command
\NewDocumentCommand\EE{m}{\texttt{\fakeverb{#1}}} % environment
\NewDocumentCommand\LL{m}{\texttt{\fakeverb{#1}}} % library
\NewDocumentCommand\PP{m}{\texttt{\fakeverb{#1}}} % package
\NewDocumentCommand\FF{m}{\texttt{\fakeverb{#1}}} % file
\NewDocumentCommand\NN{m}{\texttt{\fakeverb{#1}}} % tikz node
\NewDocumentCommand\KP{m}{\texttt{\fakeverb{#1}}} % key path
\NewDocumentCommand\HP{m}{\texttt{\fakeverb{#1}}} % hook path
\NewDocumentCommand\CI{m}{\texttt{\fakeverb{#1}}} % child indexer
\NewDocumentCommand\CS{m}{\texttt{\fakeverb{#1}}} % child selector
\NewDocumentCommand\CO{m}{\texttt{\fakeverb{#1}}} % counter
\NewDocumentCommand\EN{m}{\texttt{\fakeverb{#1}}} % element name
\NewDocumentCommand\TN{m}{\texttt{\fakeverb{#1}}} % template name
\NewDocumentCommand\TT{m}{\texttt{\fakeverb{#1}}} % text

\NewTblrEnviron{newtblr}
\SetTblrOuter[newtblr]{long}
\SetTblrInner[newtblr]{
  hlines = {gray3}, column{1,2} = {co=1}, colsep = 5pt,
  row{2-Z} = {brown8},
  row{1} = {fg=white, bg=purple2, font=\bfseries\sffamily},
}

\NewTblrEnviron{spectblr}
\SetTblrOuter[spectblr]{long}
\SetTblrInner[spectblr]{
  hlines = {gray3}, column{2} = {co=1}, colsep = 5pt,
  row{2-Z} = {brown8},
  row{1} = {fg=white, bg=purple2, font=\bfseries\sffamily},
  rowhead = 1,
}

\renewcommand\emph[1]{\textit{\color{red3}#1}}

\newcommand{\mywarning}[1]{%
  \begin{tcolorbox}
  #1
  \end{tcolorbox}%
}

%\renewcommand*{\thefootnote}{*}

\colorlet{highback}{azure9}
\CodeHigh{language=latex/table,style/main=highback,style/code=highback}
\NewCodeHighEnv{code}{style/main=gray9,style/code=gray9}
\NewCodeHighEnv{demo}{style/main=gray9,style/code=gray9,demo}

%\CodeHigh{lite}

\CodeHigh{lite}
\setcounter{chapter}{4}

\begin{document}

\chapter{Use Some Libraries}

\mywarning{chapter}

The \verb!tabularray! package emulates or fixes some commands in other packages.
To avoid potential conflict, you need to enable them with \verb!\UseTblrLibrary! command.

\section{Library \texttt{booktabs}}

When you write \verb!\UseTblrLibrary{booktabs}!,
\verb!tabularray! package will define commands \verb!\toprule!, \verb!\midrule!,
\verb!\bottomrule! and \verb!\cmidrule! inside \verb!tblr! environment.

\begin{demohigh}
\begin{tblr}{llll}
\toprule
 Alpha   & Beta  & Gamma   & Delta \\
\midrule
 Epsilon & Zeta  & Eta     & Theta \\
\cmidrule{1-3}
 Iota    & Kappa & Lambda  & Mu    \\
\cmidrule{2-4}
 Nu      & Xi    & Omicron & Pi    \\
\bottomrule
\end{tblr}
\end{demohigh}

At this moment, \verb!trim! options for \verb!\cmidrule! command are not supported.
(As a workaround, you may insert an empty column to separate two \verb!\cmidrule!'s.)
But rule colors are possible just like \verb!\hline! and \verb!\cline! commands.

\begin{demohigh}
\begin{tblr}{llll}
\toprule[purple3]
 Alpha   & Beta  & Gamma   & Delta \\
\midrule[blue3]
 Epsilon & Zeta  & Eta     & Theta \\
\cmidrule[azure3]{1-3}
 Iota    & Kappa & Lambda  & Mu    \\
\cmidrule[azure3]{2-4}
 Nu      & Xi    & Omicron & Pi    \\
\bottomrule[purple3]
\end{tblr}
\end{demohigh}

\section{Library \texttt{diagbox}}

When writing \verb!\UseTblrLibrary{diagbox}! in the preamble of the document,
\verb!tabularray! package loads \verb!diagbox! package,
and you can use \verb!\diagbox! and \verb!\diagboxthree! commands inside \verb!tblr! environment.

\begin{demohigh}
\begin{tblr}{hlines,vlines}
 \diagbox{Aa}{Pp} & Beta & Gamma \\
 Epsilon & Zeta  & Eta \\
 Iota    & Kappa & Lambda \\
\end{tblr}
\end{demohigh}

\begin{demohigh}
\begin{tblr}{hlines,vlines}
 \diagboxthree{Aa}{Pp}{Hh} & Beta & Gamma \\
 Epsilon & Zeta  & Eta \\
 Iota    & Kappa & Lambda \\
\end{tblr}
\end{demohigh}

You can also use \verb!\diagbox! and \verb!\diagboxthree! commands in math mode.
\nopagebreak
\begin{demohigh}
$\begin{tblr}{|c|cc|}
\hline
 \diagbox{X_1}{X_2} & 0 & 1 \\
\hline
  0 & 0.1 & 0.2 \\
  1 & 0.3 & 0.4 \\
\hline
\end{tblr}$
\end{demohigh}

\section{Library \texttt{siunitx}}

When writing \verb!\UseTblrLibrary{siunitx}! in the preamble of the document,
\verb!tabularray! package loads \verb!siunitx! package,
and defines \verb!S! column as \verb!Q! column with \verb!si! key.

\begin{demohigh}
\begin{tblr}{
  hlines, vlines,
  colspec={
    S[table-format=3.2]
    S[table-format=3.2]
    S[table-format=3.2]
  }
}
 {{{Head}}} & {{{Head}}} & {{{Head}}} \\
   111      &   111      &   111      \\
     2.1    &     2.2    &     2.3    \\
    33.11   &    33.22   &    33.33   \\
\end{tblr}
\end{demohigh}

\begin{demohigh}
\begin{tblr}{
  hlines, vlines,
  colspec={
    Q[si={table-format=3.2},c]
    Q[si={table-format=3.2},c]
    Q[si={table-format=3.2},c]
  }
}
 {{{Head}}} & {{{Head}}} & {{{Head}}} \\
   111      &   111      &   111      \\
     2.1    &     2.2    &     2.3    \\
    33.11   &    33.22   &    33.33   \\
\end{tblr}
\end{demohigh}

Note that you need to use \underline{triple} pairs of braces to guard non-numeric cells.

Also you must use \verb!l!, \verb!c! or \verb!r! to set horizontal alignment for non-numeric cells:

\begin{demohigh}
\begin{tblr}{
  hlines, vlines, columns={6em},
  colspec={
    Q[si={table-format=3.2,table-number-alignment=left},l,blue7]
    Q[si={table-format=3.2,table-number-alignment=center},c,teal7]
    Q[si={table-format=3.2,table-number-alignment=right},r,purple7]
  }
}
 {{{Head}}} & {{{Head}}} & {{{Head}}} \\
   111      &   111      &   111      \\
     2.1    &     2.2    &     2.3    \\
    33.11   &    33.22   &    33.33   \\
\end{tblr}
\end{demohigh}

Both \verb!S! and \verb!s! columns are supported. In fact, These two columns are defined as follows:
\begin{codehigh}
\NewColumnType{S}[1][]{Q[si={##1},c]}
\NewColumnType{s}[1][]{Q[si={##1},c,cmd=\TblrUnit]}
\end{codehigh}

\end{document}
