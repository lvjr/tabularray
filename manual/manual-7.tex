% -*- coding: utf-8 -*-
% !TEX program = lualatex
\documentclass[oneside]{book}

% -*- coding: utf-8 -*-
% !TEX program = lualatex

\newcommand*{\myversion}{2025A}
\newcommand*{\mylpad}[1]{\ifnum#1<10 0\the#1\else\the#1\fi}

\usepackage[a4paper,margin=2.5cm]{geometry}

\setlength{\parindent}{0pt}
\setlength{\parskip}{4pt plus 1pt minus 1pt}

\makeatletter
\ExplSyntaxOn
\NewDocumentCommand\MyDebugSingle{m}{
  \@ifpackageloaded{docmute}{#1}{
    \str_if_eq:onTF{\jobname}{tabularray}{#1}{
      \sys_ensure_backend:
      \debug_on:n{check-declarations}
      #1
      \debug_off:n{check-declarations}
    }
  }
}
\ExplSyntaxOff
\makeatother
\MyDebugSingle{\usepackage{tabularray}}

\usepackage{codehigh} % https://ctan.org/pkg/codehigh
\usepackage{array,multirow,amsmath}
\usepackage{chemmacros,environ}
\usepackage{enumitem}

\usepackage[firstpage=true]{background}
\backgroundsetup{contents={}}

\UseTblrLibrary{
  amsmath,booktabs,counter,diagbox,functional,siunitx,tikz,varwidth
}
\usetikzlibrary{patterns}

\usepackage{hyperref}
\hypersetup{
  colorlinks=true,
  urlcolor=blue3,
  linkcolor=blue3,
}

\usepackage{tcolorbox}
\tcbset{sharp corners, boxrule=0.5pt, colback=red9}

\usepackage{float}
%\usepackage{enumerate}

\setcounter{tocdepth}{1}

\NewDocumentCommand\None{}{{\boldmath$\times$}}
\NewDocumentCommand\K{m}{\texttt{#1}} % key
\NewDocumentCommand\V{m}{\texttt{#1}} % value
\NewDocumentCommand\Q{m}{\texttt{#1}} % column/row type

\NewDocumentCommand\KK{m}{\texttt{\fakeverb{#1}}} % key
\NewDocumentCommand\VV{m}{\texttt{\fakeverb{#1}}} % value
\NewDocumentCommand\KV{m}{\texttt{\fakeverb{#1}}} % key and value
\NewDocumentCommand\CC{m}{\texttt{\fakeverb{#1}}} % command
\NewDocumentCommand\EE{m}{\texttt{\fakeverb{#1}}} % environment
\NewDocumentCommand\LL{m}{\texttt{\fakeverb{#1}}} % library
\NewDocumentCommand\PP{m}{\texttt{\fakeverb{#1}}} % package
\NewDocumentCommand\FF{m}{\texttt{\fakeverb{#1}}} % file
\NewDocumentCommand\NN{m}{\texttt{\fakeverb{#1}}} % tikz node
\NewDocumentCommand\KP{m}{\texttt{\fakeverb{#1}}} % key path
\NewDocumentCommand\HP{m}{\texttt{\fakeverb{#1}}} % hook path
\NewDocumentCommand\CI{m}{\texttt{\fakeverb{#1}}} % child indexer
\NewDocumentCommand\CS{m}{\texttt{\fakeverb{#1}}} % child selector
\NewDocumentCommand\CO{m}{\texttt{\fakeverb{#1}}} % counter
\NewDocumentCommand\EN{m}{\texttt{\fakeverb{#1}}} % element name
\NewDocumentCommand\TN{m}{\texttt{\fakeverb{#1}}} % template name
\NewDocumentCommand\TT{m}{\texttt{\fakeverb{#1}}} % text

\NewTblrEnviron{newtblr}
\SetTblrOuter[newtblr]{long}
\SetTblrInner[newtblr]{
  hlines = {gray3}, column{1,2} = {co=1}, colsep = 5pt,
  row{2-Z} = {brown8},
  row{1} = {fg=white, bg=purple2, font=\bfseries\sffamily},
}

\NewTblrEnviron{spectblr}
\SetTblrOuter[spectblr]{long}
\SetTblrInner[spectblr]{
  hlines = {gray3}, column{2} = {co=1}, colsep = 5pt,
  row{2-Z} = {brown8},
  row{1} = {fg=white, bg=purple2, font=\bfseries\sffamily},
  rowhead = 1,
}

\renewcommand\emph[1]{\textit{\color{red3}#1}}

\newcommand{\mywarning}[1]{%
  \begin{tcolorbox}
  #1
  \end{tcolorbox}%
}

%\renewcommand*{\thefootnote}{*}

\colorlet{highback}{azure9}
\CodeHigh{language=latex/table,style/main=highback,style/code=highback}
\NewCodeHighEnv{code}{style/main=gray9,style/code=gray9}
\NewCodeHighEnv{demo}{style/main=gray9,style/code=gray9,demo}

%\CodeHigh{lite}

\CodeHigh{lite}
\setcounter{chapter}{6}

\begin{document}

\chapter{Experimental Interfaces}
\label{chap:exp}

\mywarning{chapter}

\section{Experimental Public Key Paths}

In version 2025A, all \PP{tabularray} key paths were cleaned up as follows:
\begin{itemize}[nosep]
  \item \KP{tabularray/table/inner} (from \KP{tblr})
  \item \KP{tabularray/table/outer} (from \KP{tblr-outer})
  \item \KP{tabularray/column/inner} (from \KP{tblr-column})
  \item \KP{tabularray/row/inner} (from \KP{tblr-row})
  \item \KP{tabularray/cell/inner} (from \KP{tblr-cell-spec})
  \item \KP{tabularray/cell/outer} (from \KP{tblr-cell-span})
  \item \KP{tabularray/hline/inner} (from \KP{tblr-hline})
  \item \KP{tabularray/vline/inner} (from \KP{tblr-vline})
  \item \KP{tabularray/hborder/inner} (from \KP{tblr-hborder})
  \item \KP{tabularray/vborder/inner} (from \KP{tblr-vborder})
\end{itemize}
An advanced user or package writer can use \CC{\DeclareKeys} and \CC{\SetKeys} commands
(provided by LaTeX format) to declare new keys and apply key-value lists, respectively.

The key paths are quite long, therefore \PP{tabularray} provides two shortcut commands
\CC{\DeclareTblrKeys} and \CC{\SetTblrKeys}:

\begin{codehigh}
\DeclareTblrKeys{<path>}{<keyvals>} = \DeclareKeys[tabularray/<path>]{<keyvals>}
\SetTblrKeys{<path>}{<keyvals>} = \SetKeys[tabularray/<path>]{<keyvals>}
\end{codehigh}

\section{Experimental Public Hook Names}

All experimental public \PP{tabularray} hook names provided by \LL{hook} library are as follows:
\begin{itemize}[nosep]
  \item \HP{tabularray/trial/before}
  \item \HP{tabularray/trial/after}
  \item \HP{tabularray/table/before}
  \item \HP{tabularray/table/after}
  \item \HP{tabularray/row/before}
  \item \HP{tabularray/row/after}
  \item \HP{tabularray/cell/before}
  \item \HP{tabularray/cell/after}
\end{itemize}
An advanced user or package writer can use \CC{\AddToHook} and \CC{\AddToHookNext} commands
(provided by LaTeX format) to inject code to \PP{tabularray} tables.

The hook names are quite long, therefore \PP{tabularray} provides two shortcut commands
\CC{\AddToTblrHook} and \CC{\AddToTblrHookNext}:

%\begin{codehigh}
%\AddToTblrHook{<name>}[<label>]{<code>}=\AddToHook{tabularray/<name>}[<label>]{<code>}
%\AddToTblrHookNext{<name>}{<code>}=\AddToHookNext{tabularray/<name>}{<code>}
%\end{codehigh}
\begin{codehigh}
\AddToTblrHook{<name>}{<code>} = \AddToHook{tabularray/<name>}{<code>}
\AddToTblrHookNext{<name>}{<code>} = \AddToHookNext{tabularray/<name>}{<code>}
\end{codehigh}

\section{Experimental Public Variables}

This variable is always available throughout the whole typesetting process of tables:
\begin{itemize}[nosep]
  \item \CC{\lTblrMeasuringBool}: if \PP{tabularray} is doing trial typesetting.
\end{itemize}
You need to make sure \KV{measure=vstore} to make \CC{\lTblrMeasuringBool} correct.

These variables are updated by default before building every cell:
\begin{itemize}[nosep]
  \item \CC{\lTblrCellRowSpanInt}: how many rows are spanned by current cell.
  \item \CC{\lTblrCellColSpanInt}: how many columns are spanned by current cell.
  \item \CC{\lTblrCellOmittedBool}: if current cell is spanned by another cell.
  \item \CC{\lTblrCellBackgroundTl}: background color of current cell.
\end{itemize}

These variables are updated by \LL{html} library before building every cell:
%(you need to write \CC{\UseTblrLibrary{html}} first):
\begin{itemize}[nosep]
  \item \CC{\lTblrCellAboveBorderStyleTl}
  \item \CC{\lTblrCellAboveBorderWidthDim}
  \item \CC{\lTblrCellAboveBorderColorTl}
  \item \CC{\lTblrCellBelowBorderStyleTl}
  \item \CC{\lTblrCellBelowBorderWidthDim}
  \item \CC{\lTblrCellBelowBorderColorTl}
  \item \CC{\lTblrCellLeftBorderStyleTl}
  \item \CC{\lTblrCellLeftBorderWidthDim}
  \item \CC{\lTblrCellLeftBorderColorTl}
  \item \CC{\lTblrCellRightBorderStyleTl}
  \item \CC{\lTblrCellRightBorderWidthDim}
  \item \CC{\lTblrCellRightBorderColorTl}
\end{itemize}
In the above, \TT{BorderStyle}, \TT{BorderWidth}, \TT{BorderColor} are similar to
\K{border-style}, \K{border-width}, \K{border-color} in HTML/CSS, respectively.
\TT{BorderStyle} and \TT{BorderColor} are empty by default.

\end{document}
