\documentclass{article}


%% this file will be loaded by regression-test.tex

\usepackage[b5paper,margin=1cm]{geometry}
\pagestyle{empty}

% avoid font size substitution messages from "LaTeX Font Info"
\usepackage{type1cm,type1ec}

\usepackage{tabularray}
\usepackage{array}


\UseTblrLibrary{tikz}
\usetikzlibrary{patterns}
\setlength{\parindent}{0pt}

\begin{document}

\START
\hrule\bigskip

\BEGINTEST{tikz: normal cells in a short table without a name}
\begin{tblrtikzbelow}
  \fill[pattern color=lightgray,pattern=bricks]
    (table.north east) rectangle (table.south west);
\end{tblrtikzbelow}%
\begin{tblrtikzabove}
  \fill[yellow7] (1-1.north west) rectangle (1-1.south east);
  \fill[red7] (1-3.north west) rectangle (1-3.south east);
  \fill[blue7] (1-5.north west) rectangle (1-5.south east);
  \fill[cyan7] (2-2.north west) rectangle (2-2.south east);
  \fill[teal7] (2-4.north west) rectangle (2-4.south east);
  \fill[purple7] (3-1.north west) rectangle (3-1.south east);
  \fill[green7] (3-3.north west) rectangle (3-3.south east);
  \fill[azure7] (3-5.north west) rectangle (3-5.south east);
\end{tblrtikzabove}%
\begin{tblr}{|c|c|c|[2pt]|c|c|}
\hline
  1-1 & 1-2 & 1-3 & 1-4 & 1-5 \\
\hline
  2-1 & 2-2 & 2-3 & 2-4 & 2-5 \\
\hline\hline[1pt]
  3-1 & 3-2 & 3-3 & 3-4 & 3-5 \\
\cline[2pt]{1-2}\cline[1pt]{3}\cline[3pt]{4-5}
\end{tblr}
\ENDTEST

\bigskip\hrule\bigskip

\BEGINTEST{tikz: normal cells in a tall table without a name}
\begin{tblrtikzbelow}
  \fill[pattern color=lightgray,pattern=dots]
    (table.north east) rectangle (table.south west);
\end{tblrtikzbelow}%
\begin{tblrtikzabove}
  \fill[yellow7] (1-2.north west) rectangle (1-2.south east);
  \fill[red7] (1-4.north west) rectangle (1-4.south east);
  \fill[cyan7] (2-1.north west) rectangle (2-1.south east);
  \fill[teal7] (2-3.north west) rectangle (2-3.south east);
  \fill[purple7] (2-5.north west) rectangle (2-5.south east);
  \fill[green7] (3-2.north west) rectangle (3-2.south east);
  \fill[azure7] (3-4.north west) rectangle (3-4.south east);
\end{tblrtikzabove}%
\begin{talltblr}[caption={I am tall}]{|c|c|c|[2pt]|c|c|}
\hline
  1-1 & 1-2 & 1-3 & 1-4 & 1-5 \\
\hline
  2-1 & 2-2 & 2-3 & 2-4 & 2-5 \\
\hline\hline[1pt]
  3-1 & 3-2 & 3-3 & 3-4 & 3-5 \\
\cline[2pt]{1-2}\cline[1pt]{3}\cline[3pt]{4-5}
\end{talltblr}
\ENDTEST

\bigskip\hrule\bigskip

\BEGINTEST{tikz: normal cells in a short table with a name}
\begin{tblrtikzbelow}
  \fill[pattern color=lightgray,pattern=crosshatch]
    (table.north east) rectangle (table.south west);
\end{tblrtikzbelow}%
\begin{tblrtikzabove}
  \fill[yellow7] (1-2.north west) rectangle (1-2.south east);
  \fill[red7] (1-4.north west) rectangle (1-4.south east);
  \fill[cyan7] (2-1.north west) rectangle (2-1.south east);
  \fill[teal7] (2-3.north west) rectangle (2-3.south east);
  \fill[purple7] (2-5.north west) rectangle (2-5.south east);
  \fill[green7] (3-2.north west) rectangle (3-2.south east);
  \fill[azure7] (3-4.north west) rectangle (3-4.south east);
\end{tblrtikzabove}%
\begin{tblr}{colspec={|c|c|c|[2pt]|c|c|},name=foobar}
\hline
  1-1 & 1-2 & 1-3 & 1-4 & 1-5 \\
\hline
  2-1 & 2-2 & 2-3 & 2-4 & 2-5 \\
\hline\hline[1pt]
  3-1 & 3-2 & 3-3 & 3-4 & 3-5 \\
\cline[2pt]{1-2}\cline[1pt]{3}\cline[3pt]{4-5}
\end{tblr}
\ENDTEST

\bigskip\hrule\bigskip

\BEGINTEST{tikz: normal cells in a tall table with a name}
\begin{tblrtikzbelow}
  \fill[pattern color=lightgray,pattern=checkerboard]
    (table.north east) rectangle (table.south west);
\end{tblrtikzbelow}%
\begin{tblrtikzabove}
  \fill[yellow7] (1-1.north west) rectangle (1-1.south east);
  \fill[red7] (1-3.north west) rectangle (1-3.south east);
  \fill[blue7] (1-5.north west) rectangle (1-5.south east);
  \fill[cyan7] (2-2.north west) rectangle (2-2.south east);
  \fill[teal7] (2-4.north west) rectangle (2-4.south east);
  \fill[purple7] (3-1.north west) rectangle (3-1.south east);
  \fill[green7] (3-3.north west) rectangle (3-3.south east);
  \fill[azure7] (3-5.north west) rectangle (3-5.south east);
\end{tblrtikzabove}%
\begin{talltblr}[
  caption={I am tall}
]{
  colspec={|c|c|c|[2pt]|c|c|}, name=foobaz
}
\hline
  1-1 & 1-2 & 1-3 & 1-4 & 1-5 \\
\hline
  2-1 & 2-2 & 2-3 & 2-4 & 2-5 \\
\hline\hline[1pt]
  3-1 & 3-2 & 3-3 & 3-4 & 3-5 \\
\cline[2pt]{1-2}\cline[1pt]{3}\cline[3pt]{4-5}
\end{talltblr}
\ENDTEST

\end{document}
